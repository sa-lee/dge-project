\documentclass[]{article}
\usepackage[T1]{fontenc}
\usepackage{lmodern}
\usepackage{amssymb,amsmath}
\usepackage{ifxetex,ifluatex}
\usepackage{fixltx2e} % provides \textsubscript
% use upquote if available, for straight quotes in verbatim environments
\IfFileExists{upquote.sty}{\usepackage{upquote}}{}
\ifnum 0\ifxetex 1\fi\ifluatex 1\fi=0 % if pdftex
  \usepackage[utf8]{inputenc}
\else % if luatex or xelatex
  \ifxetex
    \usepackage{mathspec}
    \usepackage{xltxtra,xunicode}
  \else
    \usepackage{fontspec}
  \fi
  \defaultfontfeatures{Mapping=tex-text,Scale=MatchLowercase}
  \newcommand{\euro}{€}
\fi
% use microtype if available
\IfFileExists{microtype.sty}{\usepackage{microtype}}{}
\usepackage[margin=1in]{geometry}
\usepackage{color}
\usepackage{fancyvrb}
\newcommand{\VerbBar}{|}
\newcommand{\VERB}{\Verb[commandchars=\\\{\}]}
\DefineVerbatimEnvironment{Highlighting}{Verbatim}{commandchars=\\\{\}}
% Add ',fontsize=\small' for more characters per line
\usepackage{framed}
\definecolor{shadecolor}{RGB}{248,248,248}
\newenvironment{Shaded}{\begin{snugshade}}{\end{snugshade}}
\newcommand{\KeywordTok}[1]{\textcolor[rgb]{0.13,0.29,0.53}{\textbf{{#1}}}}
\newcommand{\DataTypeTok}[1]{\textcolor[rgb]{0.13,0.29,0.53}{{#1}}}
\newcommand{\DecValTok}[1]{\textcolor[rgb]{0.00,0.00,0.81}{{#1}}}
\newcommand{\BaseNTok}[1]{\textcolor[rgb]{0.00,0.00,0.81}{{#1}}}
\newcommand{\FloatTok}[1]{\textcolor[rgb]{0.00,0.00,0.81}{{#1}}}
\newcommand{\CharTok}[1]{\textcolor[rgb]{0.31,0.60,0.02}{{#1}}}
\newcommand{\StringTok}[1]{\textcolor[rgb]{0.31,0.60,0.02}{{#1}}}
\newcommand{\CommentTok}[1]{\textcolor[rgb]{0.56,0.35,0.01}{\textit{{#1}}}}
\newcommand{\OtherTok}[1]{\textcolor[rgb]{0.56,0.35,0.01}{{#1}}}
\newcommand{\AlertTok}[1]{\textcolor[rgb]{0.94,0.16,0.16}{{#1}}}
\newcommand{\FunctionTok}[1]{\textcolor[rgb]{0.00,0.00,0.00}{{#1}}}
\newcommand{\RegionMarkerTok}[1]{{#1}}
\newcommand{\ErrorTok}[1]{\textbf{{#1}}}
\newcommand{\NormalTok}[1]{{#1}}
\usepackage{graphicx}
% Redefine \includegraphics so that, unless explicit options are
% given, the image width will not exceed the width of the page.
% Images get their normal width if they fit onto the page, but
% are scaled down if they would overflow the margins.
\makeatletter
\def\ScaleIfNeeded{%
  \ifdim\Gin@nat@width>\linewidth
    \linewidth
  \else
    \Gin@nat@width
  \fi
}
\makeatother
\let\Oldincludegraphics\includegraphics
{%
 \catcode`\@=11\relax%
 \gdef\includegraphics{\@ifnextchar[{\Oldincludegraphics}{\Oldincludegraphics[width=\ScaleIfNeeded]}}%
}%
\ifxetex
  \usepackage[setpagesize=false, % page size defined by xetex
              unicode=false, % unicode breaks when used with xetex
              xetex]{hyperref}
\else
  \usepackage[unicode=true]{hyperref}
\fi
\hypersetup{breaklinks=true,
            bookmarks=true,
            pdfauthor={Stuart Lee},
            pdftitle={COMP90014 Assignment 2},
            colorlinks=true,
            citecolor=blue,
            urlcolor=blue,
            linkcolor=magenta,
            pdfborder={0 0 0}}
\urlstyle{same}  % don't use monospace font for urls
\setlength{\parindent}{0pt}
\setlength{\parskip}{6pt plus 2pt minus 1pt}
\setlength{\emergencystretch}{3em}  % prevent overfull lines
\setcounter{secnumdepth}{0}

%%% Change title format to be more compact
\usepackage{titling}
\setlength{\droptitle}{-2em}
  \title{COMP90014 Assignment 2}
  \pretitle{\vspace{\droptitle}\centering\huge}
  \posttitle{\par}
  \author{Stuart Lee}
  \preauthor{\centering\large\emph}
  \postauthor{\par}
  \predate{\centering\large\emph}
  \postdate{\par}
  \date{19 October, 2014}




\begin{document}

\maketitle


Analysis of RNA-seq data from Lappalainen, Tuuli, et al (2013) to
examine which genes are differentially expressed between men and women.
The data is a sample of 40 individuals with 20 biological replicates for
each condition (male vs.~female).

\section{Part 1 - Analysis of full sequence
data}\label{part-1---analysis-of-full-sequence-data}

We begin by reading in the full set of counts data which has an average
of 20 million RNA-seq reads per sample.

\begin{Shaded}
\begin{Highlighting}[]
\NormalTok{all.data <-}\StringTok{ }\KeywordTok{read.delim}\NormalTok{(}\StringTok{"./data/assignment2data_fullcounts.tsv"}\NormalTok{, }\DataTypeTok{header =} \OtherTok{TRUE}\NormalTok{, }
    \DataTypeTok{stringsAsFactors =} \OtherTok{FALSE}\NormalTok{)}
\CommentTok{# first 40 column names correspond to biological replicate on either male or}
\CommentTok{# female. last 4 columns correspond to gene data}
\end{Highlighting}
\end{Shaded}

Next we filter the data by removing the rows where the median count is
equal to 0 for both males and females. The reason we do this is because
we want to exclude non-informative and low-expression genes. An
alternative to using the median count would be to compute the counts per
million (cpm) of each feature in each replicate. We could then exclude
the features where there is not at least 20 replicates with cpm above 1.

We then split the filtered data set into gene expression set and a gene
information data set. The former contains feature counts across the
replicates, while the latter contains annotation metadata about the
corresponding feature.

\begin{Shaded}
\begin{Highlighting}[]
\CommentTok{# column indexes for males and females}
\NormalTok{female.replicates <-}\StringTok{ }\KeywordTok{grep}\NormalTok{(}\StringTok{"_F"}\NormalTok{, }\KeywordTok{colnames}\NormalTok{(all.data))}
\NormalTok{male.replicates <-}\StringTok{ }\KeywordTok{grep}\NormalTok{(}\StringTok{"_M"}\NormalTok{, }\KeywordTok{colnames}\NormalTok{(all.data))}

\CommentTok{# create a logical vector for row medians both equal to 0 for males and}
\CommentTok{# females}
\NormalTok{low.counts <-}\StringTok{ }\KeywordTok{apply}\NormalTok{(all.data[, female.replicates], }\DecValTok{1}\NormalTok{, median) ==}\StringTok{ }\DecValTok{0} \NormalTok{&}\StringTok{ }\KeywordTok{apply}\NormalTok{(all.data[, }
    \NormalTok{male.replicates], }\DecValTok{1}\NormalTok{, median) ==}\StringTok{ }\DecValTok{0}

\NormalTok{filtered <-}\StringTok{ }\NormalTok{all.data[!low.counts, ]}

\NormalTok{expression <-}\StringTok{ }\NormalTok{filtered[, }\KeywordTok{c}\NormalTok{(female.replicates, male.replicates)]}
\NormalTok{gene.info <-}\StringTok{ }\NormalTok{filtered[, -}\KeywordTok{c}\NormalTok{(female.replicates, male.replicates)]}
\end{Highlighting}
\end{Shaded}

We can now commence the differential gene analysis using edgeR. We
construct a vector called groups, which corresponds to whether a feature
belongs to the male or female cohort. Then we construct a DGEList
object, which stores the count data.

\begin{Shaded}
\begin{Highlighting}[]
\CommentTok{# first 20 columns are female replicates (we code as 0), last 20 are male}
\CommentTok{# (we code as 1)}
\NormalTok{groups <-}\StringTok{ }\KeywordTok{as.numeric}\NormalTok{(}\KeywordTok{grepl}\NormalTok{(}\StringTok{"_M"}\NormalTok{, }\KeywordTok{colnames}\NormalTok{(expression)))}

\NormalTok{dge <-}\StringTok{ }\KeywordTok{DGEList}\NormalTok{(}\DataTypeTok{counts =} \NormalTok{expression, }\DataTypeTok{group =} \NormalTok{groups)}
\end{Highlighting}
\end{Shaded}

Next we need to normalise the counts. The reason we do this is to ensure
that our estimates of expression between the two groups are comparable
across genes and samples (that is, different biological conditions.).
The default setting for calculating normalisation factors in edgeR is to
use the trimmed mean of M values (TMM) method. This adjusts the counts
of the features by using a scaling factor on each sample relative to all
the others.

Following this we can implement multi-dimensional scaling on the
normalised data to examine the relationships between all samples. This
plot gives us an idea of how similar samples are within and between
groups.

\begin{Shaded}
\begin{Highlighting}[]
\CommentTok{# then normalise using trimmed mean}
\NormalTok{dge <-}\StringTok{ }\KeywordTok{calcNormFactors}\NormalTok{(dge)}

\CommentTok{# plot MDS to see if we get good seperation between conditions}
\KeywordTok{plotMDS}\NormalTok{(}\DataTypeTok{x =} \NormalTok{dge, }\DataTypeTok{col =} \KeywordTok{ifelse}\NormalTok{(groups ==}\StringTok{ }\DecValTok{1}\NormalTok{, }\StringTok{"blue"}\NormalTok{, }\StringTok{"green"}\NormalTok{), }\DataTypeTok{cex =} \FloatTok{0.4}\NormalTok{)}
\end{Highlighting}
\end{Shaded}

\begin{figure}[htbp]
\centering
\includegraphics{figures/normalisation-1.pdf}
\caption{MDS plot of full counts dataset}
\end{figure}

We can see the replicates cluster to their own groups and there is
separation between groups.

We then estimate the dispersion factors for the counts. The dispersion
factor calculated in \texttt{estimateCommonDisp} is the overall estimate
of the squared biological coefficient of variation across all tags. This
is the squared ratio of the standard deviation and mean in the overall
expression levels. The dispersion factor calculated in
\texttt{estimateTagwiseDisp} is the estimate of dispersion in individual
tags. The calculation of dispersion at both of these levels is necessary
to make valid inference on hypothesis tests of expression differences
(because it is an assumption of the negative binomial model, which is
used to model the count data).

\begin{Shaded}
\begin{Highlighting}[]
\CommentTok{# calculate dispersion}
\NormalTok{dge <-}\StringTok{ }\KeywordTok{estimateCommonDisp}\NormalTok{(dge)}
\NormalTok{dge <-}\StringTok{ }\KeywordTok{estimateTagwiseDisp}\NormalTok{(dge)}
\end{Highlighting}
\end{Shaded}

We are now ready to run hypothesis tests on our counts data. We
construct a function called \texttt{get\_pvalues}, which takes a DGEList
object and a data.frame of gene information and computes the hypothesis
tests at the gene level between groups. We output a data.frame which
contains the results of this analysis sorted by P-value in ascending
order and log fold-change in descending order.

\begin{Shaded}
\begin{Highlighting}[]
\NormalTok{all.results <-}\StringTok{ }\KeywordTok{get_pvalues}\NormalTok{(dge, gene.info)}
\end{Highlighting}
\end{Shaded}

We now present some results from our analysis:

The top ten differentially expressed genes (according to smallest
adjusted P-value) are:

\begin{Shaded}
\begin{Highlighting}[]
\CommentTok{# top 10 differentially expressed genes}
\NormalTok{cols.to.print <-}\StringTok{ }\KeywordTok{c}\NormalTok{(}\StringTok{"chromosome_name"}\NormalTok{, }\StringTok{"gene_name"}\NormalTok{, }\StringTok{"adjPValue"}\NormalTok{, }\StringTok{"logFC"}\NormalTok{)}
\NormalTok{top10 <-}\StringTok{ }\KeywordTok{xtable}\NormalTok{(}\KeywordTok{head}\NormalTok{(all.results[, cols.to.print], }\DecValTok{10}\NormalTok{))}
\KeywordTok{print}\NormalTok{(top10, }\DataTypeTok{include.rownames =} \OtherTok{FALSE}\NormalTok{, }\DataTypeTok{comment =} \OtherTok{FALSE}\NormalTok{)}
\end{Highlighting}
\end{Shaded}

\begin{table}[ht]
\centering
\begin{tabular}{llrr}
  \hline
chromosome\_name & gene\_name & adjPValue & logFC \\ 
  \hline
Y & RPS4Y1 & 0.00 & 10.36 \\ 
  Y & TXLNG2P & 0.00 & 10.16 \\ 
  Y & KDM5D & 0.00 & 10.01 \\ 
  Y & DDX3Y & 0.00 & 9.99 \\ 
  Y & UTY & 0.00 & 9.95 \\ 
  X & XIST & 0.00 & -9.86 \\ 
  Y & USP9Y & 0.00 & 9.84 \\ 
  Y & EIF1AY & 0.00 & 9.10 \\ 
  Y & ZFY & 0.00 & 9.02 \\ 
  Y & TTTY15 & 0.00 & 9.02 \\ 
   \hline
\end{tabular}
\end{table}

The bottom ten differentially expressed genes (according to largest
adjusted P-value) are:

\begin{Shaded}
\begin{Highlighting}[]
\CommentTok{# bottom 10 differentially expressed genes}
\NormalTok{bottom10 <-}\StringTok{ }\KeywordTok{xtable}\NormalTok{(}\KeywordTok{tail}\NormalTok{(all.results[, cols.to.print], }\DecValTok{10}\NormalTok{))}
\KeywordTok{print}\NormalTok{(bottom10, }\DataTypeTok{include.rownames =} \OtherTok{FALSE}\NormalTok{, }\DataTypeTok{comment =} \OtherTok{FALSE}\NormalTok{)}
\end{Highlighting}
\end{Shaded}

\begin{table}[ht]
\centering
\begin{tabular}{llrr}
  \hline
chromosome\_name & gene\_name & adjPValue & logFC \\ 
  \hline
16 & PPP4C & 1.00 & -0.00 \\ 
  X & PIGA & 1.00 & -0.00 \\ 
  9 & FUBP3 & 1.00 & 0.00 \\ 
  14 & EIF5 & 1.00 & -0.00 \\ 
  7 & FAM131B & 1.00 & -0.00 \\ 
  3 & TMEM41A & 1.00 & 0.00 \\ 
  14 & GSKIP & 1.00 & -0.00 \\ 
  22 & C22orf23 & 1.00 & 0.00 \\ 
  1 & KIAA0040 & 1.00 & 0.00 \\ 
  20 & CENPB & 1.00 & -0.00 \\ 
   \hline
\end{tabular}
\end{table}

We also construct a function that displays a heat map for the genes. The
rows are the top 20 differentially expressed genes, the columns are the
samples and each cell is filled with the log counts per million value.
We use a three colour scale to indicate whether the gene is being up
(red), down (blue), or stably (white) regulated in the replicate.

\begin{Shaded}
\begin{Highlighting}[]
\KeywordTok{top20_heatmap}\NormalTok{(dge, all.results)}
\end{Highlighting}
\end{Shaded}

\begin{figure}[htbp]
\centering
\includegraphics{figures/heat-all-1.pdf}
\caption{Heatmap of top 20 differentially expressed genes in full counts
data}
\end{figure}

We also construct a function to plot the histogram of raw P-values from
our analysis.

\begin{Shaded}
\begin{Highlighting}[]
\KeywordTok{pvalue_hist}\NormalTok{(all.results, }\DataTypeTok{breaks =} \DecValTok{150}\NormalTok{)}
\end{Highlighting}
\end{Shaded}

\begin{figure}[htbp]
\centering
\includegraphics{figures/pval-hist-1.pdf}
\caption{P-value histogram for full counts data}
\end{figure}

The P-values histogram resulting from multiple hypothesis tests on
continuous data are uniformly distributed between 0 and 1. The histogram
obtained for comparison of genes are approximately uniformly distributed
between 0 and 1, which produces approximately equal size bars in the
histogram. The spike at 0 are the result of the genes that are strongly
differentially expressed. However, because of multiple comparisons, the
number of true discoveries is the height of the spike at 0 relative to
the other bar heights up to 1. There is also a spike at 1 due to the due
to the discreteness of the counts data.

Finally, we construct a function to store a list of the answers to the
questions presented below (see appendix for details).

\begin{Shaded}
\begin{Highlighting}[]
\NormalTok{all.answers <-}\StringTok{ }\KeywordTok{answer_questions}\NormalTok{(all.results)}
\end{Highlighting}
\end{Shaded}

\subsubsection{Part 1: Questions}\label{part-1-questions}

\begin{enumerate}
\def\labelenumi{\arabic{enumi}.}
\item
  How many genes, in total, have P-values below 0.05? What proportion is
  this of all genes tested? There are 106 genes that have P-values below
  0.05. The proportion of this to all genes tested is 0.0043.
\item
  Out of all genes on the Y chromosome, how many have a P-value below
  0.05? There are 28 genes on the Y chromosome that have a P-value below
  0.05. The proportion of this to the total number genes on the Y
  chromosome is 0.7778.
\item
  Out of the 100 genes with the lowest P-values, how many are from the X
  chromosome? There are 22 X chromosome genes in the top 100 genes with
  the lowest P-values.
\item
  What is the log fold-change for the gene XIST? (This gene is found on
  the X chromosome and is responsible for inactivating one copy of the
  X-chromosome in women). Give your answer to four decimal places. The
  log fold-change for XIST is -9.8561.
\end{enumerate}

\section{Part 2: Analysis of subsampled
data}\label{part-2-analysis-of-subsampled-data}

We now carry out a differential gene expression analysis using a subset
of the full counts data. This time there is only an average of
approximately 1 million reads per sample. The analysis proceeds the same
as in Part 1.

\begin{Shaded}
\begin{Highlighting}[]
\CommentTok{# read in the subsampled data, unique to my student number}
\NormalTok{sub.data <-}\StringTok{ }\KeywordTok{read.delim}\NormalTok{(}\StringTok{"./data//assignment2data_subsampled_666351.tsv"}\NormalTok{, }\DataTypeTok{header =} \OtherTok{TRUE}\NormalTok{, }
    \DataTypeTok{stringsAsFactors =} \OtherTok{FALSE}\NormalTok{)}

\CommentTok{# column indexes for males and females}
\NormalTok{female.replicates <-}\StringTok{ }\KeywordTok{grep}\NormalTok{(}\StringTok{"_F"}\NormalTok{, }\KeywordTok{colnames}\NormalTok{(sub.data))}
\NormalTok{male.replicates <-}\StringTok{ }\KeywordTok{grep}\NormalTok{(}\StringTok{"_M"}\NormalTok{, }\KeywordTok{colnames}\NormalTok{(sub.data))}

\CommentTok{# filter and split the data set}
\NormalTok{low.counts.sub <-}\StringTok{ }\KeywordTok{apply}\NormalTok{(sub.data[, female.replicates], }\DecValTok{1}\NormalTok{, median) ==}\StringTok{ }\DecValTok{0} \NormalTok{&}\StringTok{ }\KeywordTok{apply}\NormalTok{(sub.data[, }
    \NormalTok{male.replicates], }\DecValTok{1}\NormalTok{, median) ==}\StringTok{ }\DecValTok{0}

\NormalTok{sub.filtered <-}\StringTok{ }\NormalTok{sub.data[!low.counts, ]}

\NormalTok{sub.expression <-}\StringTok{ }\NormalTok{sub.filtered[, }\KeywordTok{c}\NormalTok{(female.replicates, male.replicates)]}
\NormalTok{sub.geneinfo <-}\StringTok{ }\NormalTok{sub.filtered[, -}\KeywordTok{c}\NormalTok{(female.replicates, male.replicates)]}

\CommentTok{# conduct the differentail gene expression analysis}
\NormalTok{groups <-}\StringTok{ }\KeywordTok{as.numeric}\NormalTok{(}\KeywordTok{grepl}\NormalTok{(}\StringTok{"_M"}\NormalTok{, }\KeywordTok{colnames}\NormalTok{(sub.expression)))}

\NormalTok{dge.sub <-}\StringTok{ }\KeywordTok{DGEList}\NormalTok{(}\DataTypeTok{counts =} \NormalTok{sub.expression, }\DataTypeTok{group =} \NormalTok{groups)}

\CommentTok{# then normalise using trimmed mean}
\NormalTok{dge.sub <-}\StringTok{ }\KeywordTok{calcNormFactors}\NormalTok{(dge.sub)}
\end{Highlighting}
\end{Shaded}

We make an MDS plot as before.

\begin{Shaded}
\begin{Highlighting}[]
\CommentTok{# plot MDS to see if we get good seperation between conditions}
\KeywordTok{plotMDS}\NormalTok{(}\DataTypeTok{x =} \NormalTok{dge.sub, }\DataTypeTok{col =} \KeywordTok{ifelse}\NormalTok{(groups ==}\StringTok{ }\DecValTok{1}\NormalTok{, }\StringTok{"blue"}\NormalTok{, }\StringTok{"green"}\NormalTok{), }\DataTypeTok{cex =} \FloatTok{0.4}\NormalTok{)}
\end{Highlighting}
\end{Shaded}

\begin{figure}[htbp]
\centering
\includegraphics{figures/mds2-1.pdf}
\caption{MDS plot of replicates in subset data}
\end{figure}

In this MDS plot the samples in the female group are less similar to
each other compared to the results in part 1, and there appears to be an
outlier sample. There appears to be more spread in each group and there
is less separation between the two groups.

Again we calculate the dispersion factors before computing the
hypothesis tests.

\begin{Shaded}
\begin{Highlighting}[]
\CommentTok{# calculate dispersion}
\NormalTok{dge.sub <-}\StringTok{ }\KeywordTok{estimateCommonDisp}\NormalTok{(dge.sub)}
\NormalTok{dge.sub <-}\StringTok{ }\KeywordTok{estimateTagwiseDisp}\NormalTok{(dge.sub)}
\CommentTok{# compute P-values}
\NormalTok{results.sub.all <-}\StringTok{ }\KeywordTok{get_pvalues}\NormalTok{(dge.sub, sub.geneinfo)}
\end{Highlighting}
\end{Shaded}

From this analysis we get the following top 10 differentially expressed
genes:

\begin{Shaded}
\begin{Highlighting}[]
\CommentTok{# print results}
\NormalTok{top10.sub <-}\StringTok{ }\KeywordTok{xtable}\NormalTok{(}\KeywordTok{head}\NormalTok{(results.sub.all[, cols.to.print], }\DecValTok{10}\NormalTok{))}
\KeywordTok{print}\NormalTok{(top10.sub, }\DataTypeTok{include.rownames =} \OtherTok{FALSE}\NormalTok{, }\DataTypeTok{comment =} \OtherTok{FALSE}\NormalTok{)}
\end{Highlighting}
\end{Shaded}

\begin{table}[ht]
\centering
\begin{tabular}{llrr}
  \hline
chromosome\_name & gene\_name & adjPValue & logFC \\ 
  \hline
Y & RPS4Y1 & 0.00 & 9.30 \\ 
  Y & DDX3Y & 0.00 & 8.75 \\ 
  X & XIST & 0.00 & -9.53 \\ 
  Y & EIF1AY & 0.00 & 8.26 \\ 
  Y & KDM5D & 0.00 & 8.71 \\ 
  Y & TXLNG2P & 0.00 & 9.45 \\ 
  Y & USP9Y & 0.00 & 7.87 \\ 
  Y & UTY & 0.00 & 8.35 \\ 
  Y & PRKY & 0.00 & 7.22 \\ 
  Y & ZFY & 0.00 & 6.83 \\ 
   \hline
\end{tabular}
\end{table}

The bottom 10 differentially expressed genes are:

\begin{Shaded}
\begin{Highlighting}[]
\NormalTok{bottom10.sub <-}\StringTok{ }\KeywordTok{xtable}\NormalTok{(}\KeywordTok{tail}\NormalTok{(results.sub.all[, cols.to.print], }\DecValTok{10}\NormalTok{))}
\KeywordTok{print}\NormalTok{(bottom10.sub, }\DataTypeTok{include.rownames =} \OtherTok{FALSE}\NormalTok{, }\DataTypeTok{comment =} \OtherTok{FALSE}\NormalTok{)}
\end{Highlighting}
\end{Shaded}

\begin{table}[ht]
\centering
\begin{tabular}{llrr}
  \hline
chromosome\_name & gene\_name & adjPValue & logFC \\ 
  \hline
21 & AP001421.1 & 1.00 & 0.00 \\ 
  19 & CTD-2017D11.3 & 1.00 & 0.00 \\ 
  6 & AL441883.1 & 1.00 & 0.00 \\ 
  19 & AC003005.2 & 1.00 & 0.00 \\ 
  3 & LUZPP1 & 1.00 & 0.00 \\ 
  19 & AC007204.1 & 1.00 & 0.00 \\ 
  19 & AC115522.3 & 1.00 & 0.00 \\ 
  19 & AC068499.10 & 1.00 & 0.00 \\ 
  19 & AC003002.6 & 1.00 & 0.00 \\ 
  1 & AC096677.1 & 1.00 & 0.00 \\ 
   \hline
\end{tabular}
\end{table}

Finally, we plot the heat map of the top 20 differentially expressed
genes and the P-value distribution.

\begin{Shaded}
\begin{Highlighting}[]
\CommentTok{# answer questions}
\NormalTok{q.subs <-}\StringTok{ }\KeywordTok{answer_questions}\NormalTok{(results.sub.all)}
\end{Highlighting}
\end{Shaded}

\begin{Shaded}
\begin{Highlighting}[]
\CommentTok{# draw heatmap}
\KeywordTok{top20_heatmap}\NormalTok{(dge.sub, results.sub.all)}
\end{Highlighting}
\end{Shaded}

\begin{figure}[htbp]
\centering
\includegraphics{figures/sub-heat-1.pdf}
\caption{Heatmap of gene expression for subset data}
\end{figure}

From the heatmap we observe that log counts per million amongst
replicates in the female group are much less pronounced than in part 1.
However, the general pattern of expression is still present.

\begin{Shaded}
\begin{Highlighting}[]
\CommentTok{# draw P-value histogram}
\KeywordTok{pvalue_hist}\NormalTok{(results.sub.all, }\DataTypeTok{breaks =} \DecValTok{150}\NormalTok{)}
\end{Highlighting}
\end{Shaded}

\begin{figure}[htbp]
\centering
\includegraphics{figures/sub-pval-1.pdf}
\caption{Raw P-value distribution for subset data}
\end{figure}

\subsubsection{Part 2: Questions}\label{part-2-questions}

\begin{enumerate}
\def\labelenumi{\arabic{enumi}.}
\item
  How many genes, in total, have P-values below 0.05? What proportion is
  this of all genes tested? There are 44 genes that have P-values below
  0.05. The proportion of this to all genes tested is 0.0018.
\item
  Out of all genes on the Y chromosome, how many have a P-value below
  0.05? There are 15 genes on the Y chromosome that have a P-value below
  0.05. The proportion of this to the total number genes on the Y
  chromosome is 0.4167.
\item
  Out of the 100 genes with the lowest P-values, how many are from the X
  chromosome? There are 26 X chromosome genes in the top 100 genes with
  the lowest P-values.
\item
  What is the log fold-change for the gene XIST? (This gene is found on
  the X chromosome and is responsible for inactivating one copy of the
  X-chromosome in women). Give your answer to four decimal places. The
  log fold-change for XIST is -9.5347.
\end{enumerate}

\section{Part 3 - Sensitivity
analysis}\label{part-3---sensitivity-analysis}

Next we run a differential gene expression analysis by shuffling the
group memberships randomly.

\begin{Shaded}
\begin{Highlighting}[]
\CommentTok{# set seed for reproducibility}
\KeywordTok{set.seed}\NormalTok{(}\DecValTok{90014}\NormalTok{)}
\NormalTok{random.groups <-}\StringTok{ }\KeywordTok{sample}\NormalTok{(}\KeywordTok{c}\NormalTok{(}\DecValTok{0}\NormalTok{, }\DecValTok{1}\NormalTok{), }\DataTypeTok{size =} \DecValTok{40}\NormalTok{, }\DataTypeTok{replace =} \OtherTok{TRUE}\NormalTok{)}
\KeywordTok{print}\NormalTok{(random.groups)}
\end{Highlighting}
\end{Shaded}

\begin{verbatim}
##  [1] 1 0 0 0 0 0 0 1 0 1 0 0 1 1 0 1 1 0 1 0 0 0 0 0 1 1 0 1 0 1 1 1 1 0 1
## [36] 1 1 1 1 0
\end{verbatim}

\begin{Shaded}
\begin{Highlighting}[]
\NormalTok{dge.rand <-}\StringTok{ }\KeywordTok{DGEList}\NormalTok{(}\DataTypeTok{counts =} \NormalTok{sub.expression, }\DataTypeTok{group =} \NormalTok{random.groups)}
\NormalTok{dge.rand <-}\StringTok{ }\KeywordTok{calcNormFactors}\NormalTok{(dge.rand)}
\NormalTok{dge.rand <-}\StringTok{ }\KeywordTok{estimateCommonDisp}\NormalTok{(dge.rand)}
\NormalTok{dge.rand <-}\StringTok{ }\KeywordTok{estimateTagwiseDisp}\NormalTok{(dge.rand)}

\CommentTok{# compute p-values}
\NormalTok{results.rand <-}\StringTok{ }\KeywordTok{get_pvalues}\NormalTok{(dge.rand, sub.geneinfo)}
\end{Highlighting}
\end{Shaded}

For the sensitivity analysis we get the following top 10 differentially
expressed genes:

\begin{Shaded}
\begin{Highlighting}[]
\CommentTok{# print results}
\NormalTok{top10.rand <-}\StringTok{ }\KeywordTok{xtable}\NormalTok{(}\KeywordTok{head}\NormalTok{(results.rand[, cols.to.print], }\DecValTok{10}\NormalTok{))}
\KeywordTok{print}\NormalTok{(top10.rand, }\DataTypeTok{include.rownames =} \OtherTok{FALSE}\NormalTok{, }\DataTypeTok{comment =} \OtherTok{FALSE}\NormalTok{)}
\end{Highlighting}
\end{Shaded}

\begin{table}[ht]
\centering
\begin{tabular}{llrr}
  \hline
chromosome\_name & gene\_name & adjPValue & logFC \\ 
  \hline
22 & IGLV2-18 & 0.03 & -5.17 \\ 
  14 & IGHA1 & 0.03 & -3.25 \\ 
  14 & IGHA2 & 0.09 & -3.08 \\ 
  14 & IGHV2-5 & 0.20 & -2.31 \\ 
  11 & RIC3 & 0.20 & -1.08 \\ 
  14 & IGHV6-1 & 0.34 & 2.74 \\ 
  20 & DPM1 & 0.42 & 1.63 \\ 
  14 & AL928768.3 & 0.50 & -2.43 \\ 
  17 & MIR22HG & 0.50 & -0.67 \\ 
  2 & IGKV4-1 & 0.59 & 1.41 \\ 
   \hline
\end{tabular}
\end{table}

The bottom 10 differentially expressed genes are:

\begin{Shaded}
\begin{Highlighting}[]
\NormalTok{bottom10.rand <-}\StringTok{ }\KeywordTok{xtable}\NormalTok{(}\KeywordTok{tail}\NormalTok{(results.rand[, cols.to.print], }\DecValTok{10}\NormalTok{))}
\KeywordTok{print}\NormalTok{(bottom10.rand, }\DataTypeTok{include.rownames =} \OtherTok{FALSE}\NormalTok{, }\DataTypeTok{comment =} \OtherTok{FALSE}\NormalTok{)}
\end{Highlighting}
\end{Shaded}

\begin{table}[ht]
\centering
\begin{tabular}{llrr}
  \hline
chromosome\_name & gene\_name & adjPValue & logFC \\ 
  \hline
21 & AP001421.1 & 1.00 & 0.00 \\ 
  19 & CTD-2017D11.3 & 1.00 & 0.00 \\ 
  6 & AL441883.1 & 1.00 & 0.00 \\ 
  19 & AC003005.2 & 1.00 & 0.00 \\ 
  3 & LUZPP1 & 1.00 & 0.00 \\ 
  19 & AC007204.1 & 1.00 & 0.00 \\ 
  19 & AC115522.3 & 1.00 & 0.00 \\ 
  19 & AC068499.10 & 1.00 & 0.00 \\ 
  19 & AC003002.6 & 1.00 & 0.00 \\ 
  1 & AC096677.1 & 1.00 & 0.00 \\ 
   \hline
\end{tabular}
\end{table}

Finally, we present the top 20 heat map and the p-value distribution
plots.

\begin{Shaded}
\begin{Highlighting}[]
\NormalTok{rand.answers <-}\StringTok{ }\KeywordTok{answer_questions}\NormalTok{(results.rand)}
\end{Highlighting}
\end{Shaded}

\begin{Shaded}
\begin{Highlighting}[]
\KeywordTok{top20_heatmap}\NormalTok{(dge.rand, results.rand)}
\end{Highlighting}
\end{Shaded}

\begin{figure}[htbp]
\centering
\includegraphics{figures/heat-rand-1.pdf}
\caption{Heatmap of gene expression for sensitivity analysis}
\end{figure}

In this heatmap we see more cells that have near zero values of log
counts per million, compared to the heatmap obtained in part 2.

\begin{Shaded}
\begin{Highlighting}[]
\KeywordTok{pvalue_hist}\NormalTok{(results.rand, }\DataTypeTok{breaks =} \DecValTok{150}\NormalTok{)}
\end{Highlighting}
\end{Shaded}

\begin{figure}[htbp]
\centering
\includegraphics{figures/pval-rand-1.pdf}
\caption{P-value distribution for sensitivity analysis}
\end{figure}

\subsubsection{Part 3: Questions}\label{part-3-questions}

\begin{enumerate}
\def\labelenumi{\arabic{enumi}.}
\item
  How many genes, in total, have P-values below 0.05? What proportion is
  this of all genes tested? There are 2 genes that have P-values below
  0.05. The proportion of this to all genes tested is 10\^{}\{-4\}.
\item
  Out of all genes on the Y chromosome, how many have a P-value below
  0.05? There are 0 genes on the Y chromosome that have a P-value below
  0.05. The proportion of this to the total number genes on the Y
  chromosome is 0.
\item
  Out of the 100 genes with the lowest P-values, how many are from the X
  chromosome? There are 3 X chromosome genes in the top 100 genes with
  the lowest P-values.
\item
  What is the log fold-change for the gene XIST? (This gene is found on
  the X chromosome and is responsible for inactivating one copy of the
  X-chromosome in women). Give your answer to four decimal places. The
  log fold-change for XIST is -0.7549.
\end{enumerate}

\section{Part 4 - Discussion}\label{part-4---discussion}

\subsection{Results on full counts data vs.~results on subsampled counts
data}\label{results-on-full-counts-data-vs.results-on-subsampled-counts-data}

In Part 1 we performed an analysis on the entire RNA-seq data set, which
had approximately an average of 20 million reads per sample. This
differed from the analysis performed in part 2, where we had a
sub-sample of the original data set. This data set has on average 1
million reads per sample and reads were randomly selected, indicating
that the data set in part 2 has lower coverage than the full data set
used in part 1. This alters the count distribution of genes in each
replicate. This means that we have introduced additional sampling error
in part 2, decreasing the power to detect differentially expressed genes
and altering the magnitude of the log fold-change between groups.

We start by comparing the number of samples that were removed in our
filtered data set that we performed our hypothesis testing on. In part
1, we filtered 19443 features compared to the 29495 features removed on
the sub-sampled data set. Moreover, the features included in the
analysis for part 2 are a subset of the features in part 1, meaning that
our analysis in part 2 identifies different differentially expressed
genes.

By examining the MA-plots between analyses, we can see the effect that
reduced number of features and lower coverage had on our analysis. The
MA-plot is obtained in \texttt{edgeR} with the function
\texttt{plotSmear}, it is a plot of the log fold-change between males
and females (normalised expression) against the average log counts per
million over both groups. The yellow dots represent the `smear', these
are points, where counts are low in the libraries of either the male or
female groups. The top 20 significantly differentially expressed genes
(by lowest adjusted P-value) are highlighted red. A lowess trend line
has been fitted to the relationship between M and A.

\begin{Shaded}
\begin{Highlighting}[]
\KeywordTok{par}\NormalTok{(}\DataTypeTok{mfrow =} \KeywordTok{c}\NormalTok{(}\DecValTok{1}\NormalTok{, }\DecValTok{2}\NormalTok{))}
\NormalTok{de.tags.all <-}\StringTok{ }\NormalTok{all.results[}\DecValTok{1}\NormalTok{:}\DecValTok{20}\NormalTok{, }\StringTok{"id"}\NormalTok{]}
\NormalTok{de.tags.sub <-}\StringTok{ }\NormalTok{results.sub.all[}\DecValTok{1}\NormalTok{:}\DecValTok{20}\NormalTok{, }\StringTok{"id"}\NormalTok{]}
\KeywordTok{plotSmear}\NormalTok{(dge, }\DataTypeTok{de.tags =} \NormalTok{de.tags.all, }\DataTypeTok{lowess =} \OtherTok{TRUE}\NormalTok{)}
\KeywordTok{plotSmear}\NormalTok{(dge.sub, }\DataTypeTok{de.tags =} \NormalTok{de.tags.sub, }\DataTypeTok{lowess =} \OtherTok{TRUE}\NormalTok{)}
\end{Highlighting}
\end{Shaded}

\begin{figure}[htbp]
\centering
\includegraphics{figures/ma-plots-1.pdf}
\caption{Diagnostic plots for analysis. (left): MA-plot for full counts
data. (right): MA-plot for subset counts data.}
\end{figure}

There are several features of the two plots that we note. In both plots,
there doesn't appear to be any trend in the relationships between fold
change and average log counts per million, highlighting that the
normalisation procedure for both analyses has worked properly. Also
there are smear points in both plots where genes are in the top 20
differentially expressed genes, indicating that our P-value estimates
for these genes may be unreliable.

Comparing the results in the table below we observe that there are fewer
genes that have P-values below 0.05 in part 2 than in part 1. This makes
sense because the change in sequence depth in part 2, resulted in
alteration of the counts matrix, meaning our filtering processed removed
more candidate genes. The top 10 tables obtained in both analyses are
relatively similar, with both having 9 genes on the Y chromosome and
XIST have P-values below 0.05, with large log-FC. We expect that genes
with large magnitude log fold-changes (effect size) will still be
identified as differentially expressed even though there is less power.

We see no spike at 0 in the raw P-value histogram for the subset
results, indicating that we have found fewer truly differentially
expressed genes. Furthermore, due to discreteness of the counts data, we
get spikes at intermediate values.

\subsection{Results of sensitivity
analysis}\label{results-of-sensitivity-analysis}

The sensitivity analysis shuffles the group memberships before
conducting hypothesis testing. That is, we randomly label some `women'
as `men' and some `men' as `women' in our replicates and proceed with
conducting a differential gene expression analysis. In theory, there
should be no differential gene expression in the shuffled data, however
there will still be some hits due biological and technical variation.

In the sensitivity analysis, we obtain far fewer significantly
differentially expressed genes compared to the analysis in part 2. We
also have no significant hits on the Y chromosome as expected. This
means there appears to be no glaring issues with the sub-counts data.

\section{References}\label{references}

Anders, S., McCarthy, D. J., Chen, Y., Okoniewski, M., Smyth, G. K.,
Huber, W., \& Robinson, M. D. (2013). Count-based differential
expression analysis of RNA sequencing data using R and Bioconductor.
Nature Protocols, 8(9), 1765--1786. \url{doi:10.1038/nprot.2013.099}

Holmes, S., \& Martin, T. (n.d.). RNA Sequence Analysis in R: edgeR .
Web.Stanford.Edu. Retrieved October 16, 2014, from
\url{https://web.stanford.edu/class/bios221/labs/rnaseq/lab_4_rnaseq.html}

Lappalainen, Tuuli, et al. ``Transcriptome and genome sequencing
uncovers functional variation in humans.'' Nature (2013).

\section{Appendix}\label{appendix}

Further information to reproduce the output of this report.

\textbf{Functions}

\begin{Shaded}
\begin{Highlighting}[]
\NormalTok{get_pvalues}
\end{Highlighting}
\end{Shaded}

\begin{verbatim}
## function(dgObj, geneset) {
##     # function to output a sorted data frame of differentially expressed genes
##     tmp <- geneset
##     tmp$id <- rownames(geneset)
##     results <- exactTest(dgObj)
##     # obtain adjusted P-values using topTags
##     top.hits <- topTags(results, n = nrow(geneset), sort.by = "none")$table
##     top.hits$id <- rownames(top.hits)
##     names(top.hits) <- sub("FDR", "adjPValue", names(top.hits))
##     
##     # merge and then sort results by adjusted p-value and magnitude of log fold
##     # change
##     diff.genes <- merge(top.hits, tmp, all.x = TRUE, by = "id")
##     
##     diff.genes.sort <- diff.genes[order(diff.genes$adjPValue, -abs(diff.genes$logFC)), 
##         ]
##     
##     return(diff.genes.sort)
##     
## }
\end{verbatim}

\begin{Shaded}
\begin{Highlighting}[]
\NormalTok{answer_questions}
\end{Highlighting}
\end{Shaded}

\begin{verbatim}
## function(results_data) {
##     # this function returns a list with 4 slots where each slot contains a
##     # vector of solutions to the questions in the assignment spec assume the
##     # input data frame is sorted
##     
##     # q1 number of genes with P-values below 0.05
##     answers <- list()
##     signif.genes <- sum(results_data[, "adjPValue"] < 0.05)
##     prop.signif.genes <- round(signif.genes/nrow(results_data), 4)
##     answers[["q1"]] <- c(signif.genes, prop.signif.genes)
##     
##     # q2 how many genes on Y-chr are differentially expressed
##     ygenes <- results_data[, "chromosome_name"] == "Y"
##     signif.ychr <- sum(results_data[ygenes, "adjPValue"] < 0.05)
##     prop.signif.ychr <- round(signif.ychr/sum(ygenes), 4)
##     answers[["q2"]] <- c(signif.ychr, prop.signif.ychr)
##     
##     
##     # q3 top 100 genes, how many are from the X chromosome
##     top100 <- results_data[1:100, ]
##     topX <- sum(top100$chromosome_name == "X")
##     answers[["q3"]] <- topX
##     
##     # q4 log-FC for xist gene
##     xist.index <- which(results_data[, "gene_name"] == "XIST")
##     xist.logFC <- round(results_data[xist.index, "logFC"], 4)
##     answers[["q4"]] <- c(xist.index, xist.logFC)
##     
##     return(answers)
## }
\end{verbatim}

\begin{Shaded}
\begin{Highlighting}[]
\NormalTok{top20_heatmap}
\end{Highlighting}
\end{Shaded}

\begin{verbatim}
## function(dgeObj, results_data, ...) {
##     # draw a heatmap from dge object and a data frame that contains results of
##     # top hits of differentially expressed genes
##     
##     # generate a color-pallete
##     heatcols <- colorRampPalette(c("darkblue", "white", "darkred"))
##     log.cpm <- cpm(dgeObj, prior.count = 2, log = TRUE)
##     top20 <- log.cpm[results_data[1:20, "id"], ]
##     rownames(top20) <- results_data[1:20, "gene_name"]
##     heatmap(top20, Rowv = NA, Colv = NA, col = heatcols(256), margins = c(3, 
##         3), cexRow = 0.6, cexCol = 0.4, ...)
## }
\end{verbatim}

\begin{Shaded}
\begin{Highlighting}[]
\NormalTok{pvalue_hist}
\end{Highlighting}
\end{Shaded}

\begin{verbatim}
## function(results_data, ...) {
##     # draw a histogram from the dge p-values
##     hist(results_data[, "PValue"], main = "P-value distribution", xlab = "P-value", 
##         ...)
## }
\end{verbatim}

\begin{Shaded}
\begin{Highlighting}[]
\KeywordTok{sessionInfo}\NormalTok{()}
\end{Highlighting}
\end{Shaded}

\begin{verbatim}
## R version 3.1.1 (2014-07-10)
## Platform: x86_64-apple-darwin13.1.0 (64-bit)
## 
## locale:
## [1] en_AU.UTF-8/en_AU.UTF-8/en_AU.UTF-8/C/en_AU.UTF-8/en_AU.UTF-8
## 
## attached base packages:
## [1] stats     graphics  grDevices utils     datasets  methods   base     
## 
## other attached packages:
## [1] RColorBrewer_1.0-5 xtable_1.7-4       ggplot2_1.0.0     
## [4] edgeR_3.6.8        limma_3.20.9      
## 
## loaded via a namespace (and not attached):
##  [1] colorspace_1.2-4 digest_0.6.4     evaluate_0.5.5   formatR_1.0     
##  [5] grid_3.1.1       gtable_0.1.2     htmltools_0.2.6  knitr_1.7       
##  [9] MASS_7.3-35      munsell_0.4.2    plyr_1.8.1       proto_0.3-10    
## [13] Rcpp_0.11.3      reshape2_1.4     rmarkdown_0.3.3  scales_0.2.4    
## [17] stringr_0.6.2    tools_3.1.1      yaml_2.1.13
\end{verbatim}

\end{document}
